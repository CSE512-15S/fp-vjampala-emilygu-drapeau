\documentclass{chi2009}
\usepackage{times}
\usepackage{url}
\usepackage{graphics}
\usepackage{color}
\usepackage[pdftex]{hyperref}
\hypersetup{%
pdftitle={CourseRatings: A Better Course Evaluation Catalog},
pdfauthor={Ryan Drapeau, Emily Gu, Vimala Jampala},
pdfkeywords={KEYWORDS},
bookmarksnumbered,
pdfstartview={FitH},
colorlinks,
citecolor=black,
filecolor=black,
linkcolor=black,
urlcolor=black,
breaklinks=true,
}
\newcommand{\comment}[1]{}
\definecolor{Orange}{rgb}{1,0.5,0}
\newcommand{\todo}[1]{\textsf{\textbf{\textcolor{Orange}{[[#1]]}}}}

\pagenumbering{arabic}  % Arabic page numbers for submission.  Remove this line to eliminate page numbers for the camera ready copy

\begin{document}
% to make various LaTeX processors do the right thing with page size
\special{papersize=8.5in,11in}
\setlength{\paperheight}{11in}
\setlength{\paperwidth}{8.5in}
\setlength{\pdfpageheight}{\paperheight}
\setlength{\pdfpagewidth}{\paperwidth}

% use this command to override the default ACM copyright statement
% (e.g. for preprints). Remove for camera ready copy.
\toappear{Submitted to CSE 512 (Spring) as the final paper.}

\title{CourseRatings: A Better Course Evaluation Catalog}
\numberofauthors{3}
\author{
  \alignauthor Ryan Drapeau\\
    \affaddr{University of Washington}\\
    \email{drapeau@\footnotemark[1]}
  \alignauthor Emily Gu\\
    \affaddr{University of Washington}\\
    \email{emilygu@\footnotemark[1]}
  \alignauthor Vimala Jampala\\
    \affaddr{University of Washington}\\
    \email{vjampala@\thanks{@cs.washington.edu}}
}

\maketitle

\begin{abstract}
    ABSRACT HERE
\end{abstract}

\keywords{KEYWORDS DESCRIBING PAPER HERE}

\section{Introduction}

An explanation of the problem and the motivation for solving it.

\section{Related Work}

A description of previous papers related to your project.

\section{Methods}

A detailed explanation of the techniques and algorithms you used to solve the problem.

\section{Results}

The visualizations your system produces and data to help evaluate your approach. For example you may include running times, or the time users typically spend generating a visualization using your system.

\section{Discussion}

What has the audience learned from your work? What new insights or practices has your system enabled? A full blown user study is not expected, but informal observations of use that help evaluate your system are encouraged.

\section{Future Work}

A description of how your system could be extended or refined. We have read papers from a number of conferences throughout the course, but if you are having trouble figuring out how to write your paper, take a look at representative papers from the conferences listed above.

\bibliographystyle{abbrv}
\bibliography{references}

\end{document}
